\documentclass[12pt]{article}
\usepackage[margin=0.75in]{geometry}
\usepackage{amsmath, amsfonts, amssymb}
\usepackage{graphicx, hyperref}

\newcommand{\xmltag}[1]{$\tt{\langle #1 \rangle}$}

\begin{document}
\begin{titlepage}
  \vspace*{\fill}
  \begin{center}
    {\Huge Curriculum ViTeX: A Beautiful Resume-Oriented\\
           Marriage between \LaTeX\ and XML}\\[0.5cm]
    {\Large Muhammad Khan}\\[0.4cm]
    Instruction Manual
  \end{center}
  \vspace*{\fill}
\end{titlepage}
\section*{Introduction}
\subsection*{Once Upon a Time}
One day, as I was looking to update and polish my resume, I gathered about
two or three \LaTeX\ templates that I wanted to try out. After realizing
I was typing the same information over and over again, just in different
environments governed by different \emph{.cls} files, I figured there should be
a simple utility that just extracts this information from one master file and
fills it in where it should be. Given the option to search the internet for such
a program, or simply writing it myself, I of course went with the latter -
and thus Curriculum ViTeX was born.
\subsection*{Outline}
Curriculum ViTeX is written mostly in Python, and was developed and tested using
Python 3.4.3. The inputs to the program will be user-supplied XML files (more 
on that later) and the whole program will be run with the top-level script $\tt{cvitex.sh}$.
\subsection*{Contact}
I can be reached at $\tt{mhk98@cornell.edu}$. This project is hosted
on GitHub at \url{http://github.com/muhammadkhan/Curriculum-ViTeX}.
\subsection*{Prerequisites}
To fully be able to understand and use Curriculum ViTeX, you should
be well-versed in both \LaTeX\ and XML, the former a very popular
document typesetting environment and the latter a simple yet powerful
markup language. I have personally used the following tutorials to
learn these two systems in the past, and frequently consult them
as reference:
\begin{itemize}
\item \LaTeX\ tutorial: \url{http://artofproblemsolving.com/wiki/index.php/LaTeX}
\item XML tutorial: \url{http://www.tizag.com/xmlTutorial/}
\end{itemize}
\section*{Getting Started}
You can download the application off of the GitHub URL above, either by
cloning the repository or downloading the zip. The following directories and
 files are essential to running Curriculum ViTeX; make sure they are there an
intact:
\begin{itemize}
\item $\tt{cvitex.sh}$
\item $\tt{src/}$
\item $\tt{templates/}$
\end{itemize}
The latter directory is where you will find some sample templates to try the
program out, and where you will place any personal styles you would like to
the program to use.
\section*{RML: Resume Markup Language}
Resume Markup Language (RML) is a subset of XML that is employed to describe
resume skeletons. It is the master file used in Curriculum ViTeX from which
the program will fill in all necessary information in all the resume templates. 
RML files are denoted with a $\tt{.rml}$ extension. You can familiarize yourself
with the sample RML provided in $\tt{doc/samples/rml\_sample.rml}$. The
XML formatting is pretty straightforward. Here are a few key observations
to note:
\begin{enumerate}
\item The root tag is \xmltag{resume}
\item The only allowable children for the \xmltag{resume} tag are:
      \xmltag{personal}, \xmltag{education}, \xmltag{courses}, \xmltag{experiences}
      and \xmltag{skills}
\item Phone numbers can have at most up to one extension
\item Phone numbers (minus extension) must have length 10
\item The only allowable children for \xmltag{personal} are: 
      \xmltag{name}, \xmltag{phone}, \xmltag{email}, \xmltag{address} and
      \xmltag{website}
\item The \xmltag{education} tag can only have \xmltag{school} tags as
      children
\item Each \xmltag{school} \emph{must} have numerical attributes called
      ``beginning'' and ``end''
\item The \xmltag{majors} can only have \xmltag{major} tags as children
\item The \xmltag{degrees} can only have \xmltag{degree} tags as children
\item The \xmltag{gpa} tag \emph{must} have numerical attributes called
      ``value'' and ``maximum''
\item Each \xmltag{school} can only have \xmltag{name}, \xmltag{majors}, 
      \xmltag{degrees}, \xmltag{gpa} and \xmltag{comment} as children
\item The \xmltag{courses} tag can only have \xmltag{course} children
\item The \xmltag{skills} tag can only have \xmltag{skill} children, each
      of which must have a ``level'' attribute set to one of the four
      options shown in the sample file
\item The \xmltag{experiences} tag can only have \xmltag{experience}
      tags as children
\item Each \xmltag{experience} can only have \xmltag{title}, \xmltag{employer},
      \xmltag{duration} and \xmltag{description} as children tags
\end{enumerate}
Breaking any of these conventions will cause the program to complain
and you will not be able to generate your TeX resumes. However, you do
not have to include all of these tags if you would not like your
resume to have them.
\newpage
\section*{Resume Templates}
In the previous section, we went over the standards of the RML
markup language. If you follow those, then the program will be able
to successfully draw your information out of the file. Now, we focus
on creating the actual \LaTeX\ templates that this information will
fill. A smple is provided in $\tt{doc/samples/sample\_structure.xml}$.
The most important rule for templates is this:
$$\text{For 'mytemplate' to qualify as as a valid template,}$$
$$\text{you must save the XML file as}$$
$$\tt{templates/mytemplate/mytemplate\_structure.xml}$$
Furthermore, any relevant \textit{.cls} and \textit{.sty} files must also
reside in the same directory as the XML file. \textbf{If there is a .cls file, its name must be the same as its directory.}
In our example, the \textit{.cls} file name would have to be $\tt{mytemplate.cls}$.\\ \\
As with RML, any $\tt{*\_structure.xml}$ file must be properly formatted.
The format can be easily understood from looking at the sample.
\section*{Running the Program}
As mentioned previously, the program can be run using the $\tt{cvitex.sh}$ script. In a terminal (command prompt), navigate to the program's
top-level directory. A run of the program constitutes executing the
following type of statement:
$$\tt{./cvitex.sh\ /path/to/RML/file/x.rml\ t\_1\ t\_2\ \cdots\ t\_n}$$
where each of the arguments following the RML file corresponds to a
template. For each $\tt{t\_i}$, the program assumes:
\begin{itemize}
\item $\tt{templates/t\_i/}$ exists
\item $\tt{templates/t\_i/t\_i\_structure.xml}$ exists
\item If there is any \textit{.cls} file in $\tt{templates/t\_i/}$
      then it is named $\tt{t\_i.cls}$
\end{itemize}
If everything goes smoothly, then you will find a fully-formatted
\LaTeX\ document in the template's directory with the suffix
$\tt{\_cv.tex}$. Feel free to make further edits to this file for
particular fine-tuning, or compile as-is (tested to compile using
$\tt{pdflatex}$).
\subsection*{Sample Run}
Here is an example run of the program using the sample RML file
and one of the provided templates: $\tt{tccv}$. The statement
to execute would be (provided you are in the top-level directory):
$$\tt{./cvitex.sh\ doc/samples/rml\_sample.rml\ tccv}$$
If this runs successfully, you should find a file located at
$\tt{templates/tccv/tccv\_cv.tex}$.
\end{document}
